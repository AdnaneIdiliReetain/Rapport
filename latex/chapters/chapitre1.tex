% Chapitre 1: Contexte et Problématique

\chapter{Contexte et Problématique}

\section{Contexte du Projet}
Dans cette section, nous présentons le contexte général du projet, l'environnement dans lequel il s'inscrit, et les motivations qui ont conduit à sa réalisation.

\subsection{Présentation de l'Entreprise}
[Description détaillée de l'entreprise d'accueil, son secteur d'activité, ses produits/services, etc.]

\subsection{Cadre du Projet}
[Explication du cadre dans lequel s'inscrit ce projet au sein de l'entreprise]

\section{Problématique}
Cette section expose la problématique à laquelle répond ce projet de fin d'études.

\subsection{Analyse de l'Existant}
[Analyse de la situation actuelle et des limitations qui justifient le projet]

\subsection{Enjeux et Défis}
[Présentation des enjeux stratégiques et des défis techniques à relever]

\section{Objectifs du Projet}
Cette section détaille les objectifs spécifiques du projet.

\subsection{Objectifs Généraux}
[Description des objectifs généraux du projet]

\subsection{Objectifs Spécifiques}
[Énumération et explication des objectifs spécifiques à atteindre]

\section{Méthodologie Adoptée}
Cette section présente l'approche méthodologique adoptée pour mener à bien ce projet.

\subsection{Choix de la Méthodologie}
[Justification du choix de la méthodologie utilisée]

\subsection{Phases du Projet}
[Description des différentes phases prévues pour la réalisation du projet]

\section{Conclusion}
[Conclusion du chapitre résumant les points clés et faisant la transition vers le chapitre suivant]
