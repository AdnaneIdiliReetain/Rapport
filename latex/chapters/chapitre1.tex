% Chapitre 1: Contexte Organisationnel et Technique

\chapter{Contexte Organisationnel et Technique}

\section{Présentation de Reetain Consulting}
\begin{mdframed}[backgroundcolor=lightgreen!20, linewidth=1pt]
Reetain Consulting est une société de conseil spécialisée dans l'implémentation de solutions CRM et l'optimisation de l'expérience client pour les marques de luxe et de mode. Avec une expertise particulière dans l'écosystème Salesforce, Reetain accompagne ses clients dans leur transformation digitale et l'unification de leurs données clients.
\end{mdframed}

\subsection{Expertise et Services}
Reetain Consulting offre une gamme complète de services incluant :

\begin{center}
\begin{tabular}{|>{
\bfseries}p{4cm}|p{9.5cm}|}
\hline
\rowcolor{lightblue} Domaine d'Expertise & Description \\
\hline
Intégration CRM & Implémentation et personnalisation de solutions Salesforce adaptées aux besoins spécifiques des marques de luxe \\
\hline
Stratégie Omnicanale & Conception et mise en œuvre de stratégies d'unification des canaux de vente et de communication \\
\hline
Gestion des Données & Consolidation, nettoyage et enrichissement des données clients provenant de sources multiples \\
\hline
Automatisation Marketing & Développement de parcours client automatisés et personnalisés \\
\hline
\end{tabular}
\end{center}

\section{Présentation d'AMI Paris}
AMI Paris est une marque de mode française fondée en 2011 par Alexandre Mattiussi. Positionnée sur le segment du luxe accessible, AMI propose des collections pour homme et femme qui allient élégance parisienne et décontraction contemporaine.

\subsection{Écosystème Commercial}
La marque opère à travers plusieurs canaux de distribution :

\begin{itemize}
    \item \textbf{Réseau de boutiques physiques} : 12 boutiques en propre réparties dans les capitales mondiales de la mode
    \item \textbf{E-commerce} : Plateforme Shopify représentant 35\% du chiffre d'affaires global
    \item \textbf{Distribution wholesale} : Présence dans plus de 360 points de vente multimarques
    \item \textbf{Marketplaces} : Partenariats avec des plateformes de luxe comme Farfetch et Net-a-Porter
\end{itemize}

\section{Analyse de l'Existant}

\subsection{Systèmes d'Information Actuels}
Avant le projet AMIgo, l'écosystème technique d'AMI Paris présentait une architecture fragmentée :

\begin{center}
\begin{tabular}{|>{
\bfseries}p{3.5cm}|p{5cm}|p{5cm}|}
\hline
\rowcolor{lightblue} Système & Fonction & Limitations \\
\hline
Cegid Y2 & Gestion des points de vente physiques et des clients retail & Données isolées, pas d'intégration native avec l'e-commerce \\
\hline
Shopify & Plateforme e-commerce et gestion des clients online & Base de données clients séparée de Cegid Y2 \\
\hline
Mailchimp & Campagnes email marketing & Synchronisation manuelle des listes, segmentation limitée \\
\hline
Tableur Excel & Réconciliation manuelle des données clients & Processus chronophage et sujet aux erreurs \\
\hline
\end{tabular}
\end{center}

\subsection{Défis de l'Intégration Multicanale}
\begin{mdframed}[backgroundcolor=lightgreen!20, linewidth=1pt]
L'absence d'une vision client unifiée engendre plusieurs problématiques critiques pour AMI Paris :
\begin{itemize}
    \item Impossibilité d'identifier un même client à travers les différents canaux
    \item Duplication des profils clients et des communications
    \item Expérience client incohérente entre les canaux physiques et digitaux
    \item Difficulté à mesurer la valeur client globale et le ROI des actions marketing
    \item Complexité dans la gestion des consentements RGPD à travers les différents systèmes
\end{itemize}
\end{mdframed}

\section{Enjeux Stratégiques du Projet}
Le projet AMIgo Client Engagement s'inscrit dans une stratégie plus large de transformation digitale et d'amélioration de l'expérience client chez AMI Paris.

\subsection{Objectifs Business}
\begin{itemize}
    \item Augmenter le taux de conversion et le panier moyen grâce à une meilleure connaissance client
    \item Améliorer la fidélisation client par des communications personnalisées et pertinentes
    \item Optimiser l'allocation des ressources marketing grâce à une meilleure segmentation
    \item Renforcer la cohérence de l'expérience de marque à travers tous les points de contact
\end{itemize}

\subsection{Indicateurs de Performance Clés}
\begin{center}
\begin{tabular}{|>{
\bfseries}p{5cm}|p{3cm}|p{5cm}|}
\hline
\rowcolor{lightblue} KPI & Objectif & Méthode de Mesure \\
\hline
Taux de conversion omnicanal & +25\% & Suivi des parcours clients cross-canal \\
\hline
Valeur Vie Client (LTV) & +30\% & Analyse de cohortes pré/post implémentation \\
\hline
Taux d'engagement email & +40\% & Métriques d'ouverture et de clic \\
\hline
Efficacité opérationnelle & -50\% de temps & Réduction du temps consacré à la réconciliation des données \\
\hline
\end{tabular}
\end{center}

\section{Méthodologie Adoptée}
Cette section présente l'approche méthodologique adoptée pour mener à bien ce projet.

\subsection{Choix de la Méthodologie}
[Justification du choix de la méthodologie utilisée]

\subsection{Phases du Projet}
[Description des différentes phases prévues pour la réalisation du projet]

\section{Conclusion}
[Conclusion du chapitre résumant les points clés et faisant la transition vers le chapitre suivant]
