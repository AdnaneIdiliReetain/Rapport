% Chapitre 3: Référentiel Client Unifié (RCU)

\chapter{Référentiel Client Unifié (RCU)}

\section{Conception du Modèle de Données Client}

\begin{mdframed}[backgroundcolor=lightgreen!20, linewidth=1pt]
Le Référentiel Client Unifié (RCU) constitue le cœur du projet AMIgo, permettant de centraliser et d'harmoniser toutes les données clients provenant des différents canaux. Sa conception a nécessité une analyse approfondie des besoins métier et des contraintes techniques pour garantir une vision client à 360°.
\end{mdframed}

\subsection{Architecture du Modèle de Données}

L'architecture du RCU repose sur plusieurs objets Salesforce interconnectés :

\begin{center}
\begin{tabular}{|>{\bfseries}p{4cm}|p{9.5cm}|}
\hline
\rowcolor{lightblue} Objet & Description \\
\hline
Account & Entité centrale représentant le client avec ses informations d'identification et de contact \\
\hline
Store & Représentation des boutiques physiques avec leurs caractéristiques et zones de chalandise \\
\hline
Sales Associate & Profils des conseillers de vente avec leurs attributions et historiques de performance \\
\hline
Consent & Gestion des consentements RGPD avec historisation des modifications \\
\hline
Order & Historique des achats cross-canal avec détails des transactions \\
\hline
\end{tabular}
\end{center}

\subsection{Modélisation des Relations}

Les relations entre les différents objets ont été conçues pour faciliter les requêtes métier tout en maintenant l'intégrité des données :

\begin{itemize}
    \item Relation 1-n entre Account et Order pour l'historique d'achat
    \item Relation n-n entre Account et Store via un objet de jonction pour suivre les préférences de boutique
    \item Relation n-n entre Account et Sales Associate pour la gestion de la clientèle
    \item Relation 1-n entre Account et Consent pour le suivi des consentements marketing
\end{itemize}

\section{CRM RCU - Account Object (ACE-24)}

L'objet Account est la pierre angulaire du RCU, centralisant toutes les informations clients essentielles.

\subsection{Structure de l'Objet Account}

\begin{center}
\begin{tabular}{|>{\bfseries}p{4cm}|p{4cm}|p{5.5cm}|}
\hline
\rowcolor{lightblue} Champ & Type & Description \\
\hline
Customer ID & Texte (External ID) & Identifiant unique du client généré par le système \\
\hline
Email & Email & Adresse email principale du client (unique) \\
\hline
First Name & Texte & Prénom du client \\
\hline
Last Name & Texte & Nom de famille du client \\
\hline
Phone & Téléphone & Numéro de téléphone principal \\
\hline
Preferred Language & Liste de sélection & Langue de communication préférée \\
\hline
Preferred Store & Référence & Boutique préférée du client \\
\hline
Customer Origin & Liste de sélection & Canal d'acquisition (Retail, E-commerce, Wholesale) \\
\hline
Customer Tier & Liste de sélection & Segment client (Standard, Premium, VIP) \\
\hline
\end{tabular}
\end{center}

\subsection{Mécanismes de Déduplication}

\begin{mdframed}[backgroundcolor=lightgreen!20, linewidth=1pt]
La déduplication des profils clients est un enjeu critique pour le RCU. Plusieurs mécanismes ont été implémentés :

\begin{itemize}
    \item Algorithme de matching basé sur l'email, le téléphone et l'adresse postale
    \item Système de scoring de similarité pour les cas ambigus
    \item Processus de validation manuelle pour les cas complexes
    \item Historisation des fusions pour traçabilité
\end{itemize}
\end{mdframed}

\section{CRM RCU - Stores (ACE-10)}

L'objet Store permet de gérer le réseau de boutiques physiques et leurs interactions avec les clients.

\subsection{Structure de l'Objet Store}

\begin{center}
\begin{tabular}{|>{\bfseries}p{4cm}|p{4cm}|p{5.5cm}|}
\hline
\rowcolor{lightblue} Champ & Type & Description \\
\hline
Store ID & Texte (External ID) & Identifiant unique de la boutique \\
\hline
Store Name & Texte & Nom commercial de la boutique \\
\hline
Address & Adresse & Adresse physique complète \\
\hline
Phone & Téléphone & Numéro de contact de la boutique \\
\hline
Email & Email & Adresse email de la boutique \\
\hline
Opening Hours & Texte long & Horaires d'ouverture \\
\hline
Store Manager & Référence & Responsable de la boutique \\
\hline
Region & Liste de sélection & Zone géographique (Europe, Asie, Amériques) \\
\hline
\end{tabular}
\end{center}

\section{CRM RCU - GDPR Process (ACE-13)}

La conformité RGPD est un aspect fondamental du projet AMIgo, nécessitant des mécanismes robustes de gestion des consentements et des droits des clients.

\subsection{Modèle de Consentement}

Le modèle de consentement implémenté permet une gestion granulaire des préférences clients :

\begin{center}
\begin{tabular}{|>{\bfseries}p{4cm}|p{9.5cm}|}
\hline
\rowcolor{lightblue} Type de Consentement & Description \\
\hline
Email Marketing & Consentement pour recevoir des communications marketing par email \\
\hline
SMS Marketing & Consentement pour recevoir des communications marketing par SMS \\
\hline
Téléphone & Consentement pour être contacté par téléphone \\
\hline
Courrier Postal & Consentement pour recevoir des communications par voie postale \\
\hline
Analyse des Données & Consentement pour l'analyse comportementale et la personnalisation \\
\hline
\end{tabular}
\end{center}

\subsection{Processus de Gestion des Droits}

\begin{mdframed}[backgroundcolor=lightgreen!20, linewidth=1pt]
Un workflow automatisé a été implémenté pour traiter les demandes relatives aux droits RGPD :

\begin{itemize}
    \item Droit d'accès : génération automatique d'un rapport des données personnelles
    \item Droit de rectification : interface dédiée pour la mise à jour des informations
    \item Droit à l'oubli : processus d'anonymisation avec validation multi-niveaux
    \item Droit d'opposition : mécanisme de désabonnement avec confirmation
    \item Droit à la portabilité : export des données au format standard
\end{itemize}
\end{mdframed}

\section{CRM RCU - Data Merge Process (ACE-44)}

Le processus de fusion des données est crucial pour maintenir l'intégrité du RCU lors de l'identification de doublons.

\subsection{Algorithme de Fusion}

L'algorithme de fusion implémenté suit une logique de priorisation des sources de données :

\begin{center}
\begin{tabular}{|>{\bfseries}p{3cm}|p{3cm}|p{7.5cm}|}
\hline
\rowcolor{lightblue} Source & Priorité & Justification \\
\hline
Cegid Y2 & Haute & Données collectées en face-à-face avec vérification d'identité \\
\hline
Shopify & Moyenne & Données déclaratives avec validation d'email \\
\hline
Formulaires Web & Basse & Données non vérifiées, potentiellement incomplètes \\
\hline
\end{tabular}
\end{center}

\subsection{Traçabilité des Fusions}

Chaque fusion de profils clients est documentée dans un journal d'audit contenant :

\begin{itemize}
    \item Identifiants des profils source et cible
    \item Horodatage de la fusion
    \item Utilisateur ou processus ayant initié la fusion
    \item Règles de décision appliquées
    \item Champs modifiés avec valeurs avant/après
\end{itemize}

\section{Résultats et Défis}

\subsection{Métriques de Qualité des Données}

\begin{center}
\begin{tabular}{|>{\bfseries}p{5cm}|p{3cm}|p{5cm}|}
\hline
\rowcolor{lightblue} Indicateur & Résultat & Impact Business \\
\hline
Taux de déduplication & 18\% & Réduction des communications en double \\
\hline
Complétude des profils & +45\% & Amélioration de la segmentation client \\
\hline
Précision des données & 92\% & Fiabilité accrue des analyses \\
\hline
Temps de réconciliation & -85\% & Gain de productivité pour les équipes \\
\hline
\end{tabular}
\end{center}

\subsection{Défis Rencontrés et Solutions}

\begin{mdframed}[backgroundcolor=lightgreen!20, linewidth=1pt]
Plusieurs défis techniques et organisationnels ont été surmontés durant l'implémentation du RCU :

\begin{itemize}
    \item \textbf{Normalisation des données} : Développement d'un framework de standardisation des formats d'adresse et de téléphone
    \item \textbf{Gestion des homonymes} : Implémentation d'un système de scoring multi-critères pour différencier les homonymes
    \item \textbf{Résistance au changement} : Programme de formation et d'accompagnement des équipes retail
    \item \textbf{Performance du système} : Optimisation des requêtes et mise en place d'une architecture de cache
\end{itemize}
\end{mdframed}
