% Chapitre 4: Intégration et Chargement des Données

\chapter{Intégration et Chargement des Données}

\section{Initial Data Load Framework (ACE-5)}

\begin{mdframed}[backgroundcolor=lightgreen!20, linewidth=1pt]
Le chargement initial des données historiques constitue une étape critique du projet AMIgo, nécessitant une approche méthodique pour garantir l'intégrité et la qualité des données migrées vers le Référentiel Client Unifié.
\end{mdframed}

\subsection{Stratégie de Migration}

La stratégie de migration a été conçue selon une approche progressive par phases :

\begin{center}
\begin{tabular}{|>{\bfseries}p{3cm}|p{5cm}|p{5.5cm}|}
\hline
\rowcolor{lightblue} Phase & Périmètre & Approche \\
\hline
Phase 1 & Données clients Cegid Y2 & Migration complète avec déduplication préalable \\
\hline
Phase 2 & Données clients Shopify & Réconciliation avec les profils existants et création des nouveaux profils \\
\hline
Phase 3 & Historique des achats & Rattachement aux profils clients unifiés \\
\hline
Phase 4 & Données de consentement & Consolidation selon la règle du consentement le plus restrictif \\
\hline
\end{tabular}
\end{center}

\subsection{Framework ETL Personnalisé}

Un framework ETL (Extract, Transform, Load) sur mesure a été développé pour orchestrer le processus de migration :

\begin{itemize}
    \item \textbf{Extraction} : Connecteurs spécifiques pour Cegid Y2 et Shopify avec pagination et gestion des timeouts
    \item \textbf{Transformation} : Pipeline de nettoyage, normalisation et enrichissement des données
    \item \textbf{Chargement} : Mécanisme de chargement par lots avec validation et journalisation
    \item \textbf{Réconciliation} : Algorithmes de matching pour identifier les doublons inter-systèmes
\end{itemize}

\section{Initial Integration Preparation (ACE-4)}

La préparation de l'intégration initiale a impliqué la mise en place des fondations techniques nécessaires à l'interopérabilité des systèmes.

\subsection{Architecture d'Intégration}

\begin{center}
\begin{tabular}{|>{\bfseries}p{4cm}|p{9.5cm}|}
\hline
\rowcolor{lightblue} Composant & Fonction \\
\hline
API Gateway & Point d'entrée unique sécurisé pour toutes les intégrations \\
\hline
MuleSoft ESB & Orchestration des flux de données entre les systèmes \\
\hline
File d'attente & Gestion asynchrone des messages pour garantir la résilience \\
\hline
Service de transformation & Conversion des formats de données entre systèmes \\
\hline
Monitoring & Supervision en temps réel des flux d'intégration \\
\hline
\end{tabular}
\end{center}

\subsection{Sécurisation des Échanges}

\begin{mdframed}[backgroundcolor=lightgreen!20, linewidth=1pt]
La sécurité des données a été une priorité absolue dans la conception de l'architecture d'intégration :

\begin{itemize}
    \item Authentification mutuelle TLS pour tous les échanges inter-systèmes
    \item Chiffrement des données sensibles en transit et au repos
    \item Tokenisation des identifiants clients dans les échanges
    \item Journalisation exhaustive des accès et modifications
    \item Mécanismes de détection d'intrusion et d'anomalies
\end{itemize}
\end{mdframed}

\section{CRM - Data Cloud (ACE-69)}

Salesforce Data Cloud a été implémenté comme couche d'unification et d'enrichissement des données client.

\subsection{Fonctionnalités Implémentées}

\begin{center}
\begin{tabular}{|>{\bfseries}p{4cm}|p{9.5cm}|}
\hline
\rowcolor{lightblue} Fonctionnalité & Description \\
\hline
Identity Resolution & Réconciliation des identités clients à travers les canaux \\
\hline
Data Harmonization & Standardisation et enrichissement automatique des données \\
\hline
Segmentation Avancée & Création de segments dynamiques basés sur comportements et attributs \\
\hline
Activation Omnicanale & Distribution des données client vers les canaux d'activation \\
\hline
Insights en Temps Réel & Génération d'insights actionnables sur les comportements clients \\
\hline
\end{tabular}
\end{center}

\subsection{Modèle de Données Unifié}

Le modèle de données dans Data Cloud a été conçu pour offrir une vue client enrichie :

\begin{itemize}
    \item \textbf{Profil Unifié} : Consolidation des attributs clients provenant de toutes les sources
    \item \textbf{Graphe d'Identité} : Cartographie des identifiants clients à travers les canaux
    \item \textbf{Timeline Comportementale} : Historique chronologique des interactions client
    \item \textbf{Scores Prédictifs} : Indicateurs calculés de propension à l'achat, risque d'attrition, etc.
    \item \textbf{Affinités} : Préférences produits et catégories déduites du comportement
\end{itemize}

\section{CRM - Orders \& Line Items (ACE-70)}

L'intégration des commandes et des lignes de commande permet une vision complète de l'historique d'achat client.

\subsection{Structure des Objets de Transaction}

\begin{center}
\begin{tabular}{|>{\bfseries}p{3.5cm}|p{4cm}|p{6cm}|}
\hline
\rowcolor{lightblue} Objet & Relation & Attributs Clés \\
\hline
Order & Parent de Line Item, enfant d'Account & Date, montant total, canal, devise, statut \\
\hline
Order Line Item & Enfant d'Order, lié à Product & Produit, quantité, prix unitaire, remise \\
\hline
Product & Référencé par Line Item & SKU, catégorie, collection, saison \\
\hline
Payment & Enfant d'Order & Méthode, montant, statut, date \\
\hline
\end{tabular}
\end{center}

\subsection{Synchronisation Bidirectionnelle}

\begin{mdframed}[backgroundcolor=lightgreen!20, linewidth=1pt]
La synchronisation des commandes a été implémentée de manière bidirectionnelle :

\begin{itemize}
    \item \textbf{Cegid Y2 → Salesforce} : Synchronisation en temps réel des achats en boutique
    \item \textbf{Shopify → Salesforce} : Intégration des commandes e-commerce via webhooks
    \item \textbf{Salesforce → Cegid Y2} : Enrichissement des profils clients en boutique
    \item \textbf{Salesforce → Shopify} : Mise à jour des préférences et consentements
\end{itemize}
\end{mdframed}

\section{Défis Techniques et Résolution}

\subsection{Gestion des Volumes de Données}

Le projet a dû faire face à des volumes importants de données historiques :

\begin{center}
\begin{tabular}{|>{\bfseries}p{4cm}|p{3cm}|p{6.5cm}|}
\hline
\rowcolor{lightblue} Type de Données & Volume & Solution Implémentée \\
\hline
Profils clients & 1.2M & Chargement par lots avec fenêtres de maintenance \\
\hline
Transactions & 3.5M & Migration incrémentale par tranches temporelles \\
\hline
Interactions & 8.7M & Archivage des données anciennes avec rétention sélective \\
\hline
Produits & 45K & Synchronisation complète avec gestion des versions \\
\hline
\end{tabular}
\end{center}

\subsection{Résolution des Problèmes d'Intégration}

\begin{mdframed}[backgroundcolor=lightgreen!20, linewidth=1pt]
Plusieurs défis d'intégration ont été surmontés durant les sprints 6-7 :

\begin{itemize}
    \item \textbf{Problème} : Incohérences dans les formats de date entre systèmes\\
      \textbf{Solution} : Normalisation systématique en UTC avec conversion contextuelle
    \item \textbf{Problème} : Latence élevée lors des pics de charge\\
      \textbf{Solution} : Implémentation d'un système de file d'attente prioritaire
    \item \textbf{Problème} : Pertes de connexion avec Cegid Y2\\
      \textbf{Solution} : Mécanisme de retry exponentiel avec circuit breaker
    \item \textbf{Problème} : Doublons de transactions lors des synchronisations\\
      \textbf{Solution} : Système d'idempotence basé sur des identifiants composites
\end{itemize}
\end{mdframed}
