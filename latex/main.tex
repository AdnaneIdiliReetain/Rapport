% PFE Report LaTeX Template - AMIgo Client Engagement
% Based on Ayoub's template requirements

\documentclass[12pt,a4paper]{report}

% Required packages
\usepackage[utf8]{inputenc}
\usepackage[T1]{fontenc}
\usepackage{natbib} % Load natbib before babel/french as warned
\usepackage[french]{babel}
\usepackage{graphicx}
\usepackage{xcolor}
\usepackage{colortbl} % Required for \rowcolor
\usepackage{geometry}
\usepackage{titlesec}
\usepackage{fancyhdr}
\usepackage{tocloft}
\usepackage{enumitem}
\usepackage{tabularx}
\usepackage{booktabs}
\usepackage{hyperref}
\usepackage{pifont}
\usepackage{etoolbox}
\usepackage{setspace}
\usepackage{caption}
\usepackage{subcaption}
\usepackage{float}
\usepackage{listings}
\usepackage{appendix}
\usepackage{titletoc}
\usepackage{fancybox}
\usepackage{array}
\usepackage{tikz}
\usetikzlibrary{shadows} % Required for drop shadow
\usepackage{mdframed}
\usepackage[absolute]{textpos} % For absolute positioning

% Page geometry
\geometry{
    a4paper,
    top=2.5cm,
    bottom=2.5cm,
    left=2.5cm,
    right=2.5cm,
    headheight=14pt
}

% Define colors
\definecolor{coverblue}{RGB}{0, 51, 153} % Dark blue for titles (AMI blue)
\definecolor{lightgray}{RGB}{240, 240, 240}
\definecolor{lightblue}{RGB}{235, 245, 255}
\definecolor{lightgreen}{RGB}{235, 255, 235} % Light green for highlighted sections
\definecolor{warningorange}{RGB}{255, 240, 220}
\definecolor{notegreen}{RGB}{240, 255, 240}

% Configure hyperref
\hypersetup{
    colorlinks=true,
    linkcolor=coverblue,
    filecolor=magenta,
    urlcolor=coverblue,
    citecolor=coverblue,
    pdftitle={AMIgo Client Engagement: Référentiel Client Unifié \& Automatisations Marketing},
    pdfauthor={ÉTUDIANT PFE},
    pdfsubject={Projet de Fin d'Études - AMI Paris},
    pdfkeywords={PFE, CRM, Salesforce, AMI Paris, Omnicanal, Marketing Automation}
}

% Define a hollow circular bullet point
\newcommand{\circularbullet}{\textcolor{coverblue}{$\circ$}}

% Configure chapter and section formatting
\titleformat{\chapter}[display]
{\normalfont\huge\bfseries\color{coverblue}}
{\chaptertitlename\ \thechapter}{20pt}{\Huge}
\titlespacing*{\chapter}{0pt}{50pt}{40pt}

\titleformat{\section}
{\normalfont\Large\bfseries\color{coverblue}}
{\thesection}{1em}{}
\titlespacing*{\section}{0pt}{3.5ex plus 1ex minus .2ex}{2.3ex plus .2ex}

\titleformat{\subsection}
{\normalfont\large\bfseries}
{\thesubsection}{1em}{}
\titlespacing*{\subsection}{0pt}{3.25ex plus 1ex minus .2ex}{1.5ex plus .2ex}

% Configure headers and footers
\pagestyle{fancy}
\fancyhf{}
\fancyhead[L]{\leftmark}
\fancyhead[R]{\thepage}
\renewcommand{\headrulewidth}{0.4pt}
\renewcommand{\footrulewidth}{0pt}

% Configure TOC, LOF, LOT
\renewcommand{\cfttoctitlefont}{\huge\bfseries\color{coverblue}}
\renewcommand{\cftloftitlefont}{\huge\bfseries\color{coverblue}}
\renewcommand{\cftlottitlefont}{\huge\bfseries\color{coverblue}}

% Roman numerals for front matter
\pagenumbering{roman}

% Begin document
\begin{document}

% Cover page
\begin{titlepage}
\begin{center}
    % University and company logos with absolute positioning
    \vspace*{1cm}
    % Set the unit to cm for textpos
    \setlength{\TPHorizModule}{1cm}
    \setlength{\TPVertModule}{1cm}
    
    % Position the logos with textpos
    \begin{textblock}{5}(1.5,1)  % width of 5cm, positioned at (2,1)
        \includegraphics[height=4cm]{../images/logos/logo_school.png}
    \end{textblock}
    
    \begin{textblock}{5}(14.5,3.2)  % width of 5cm, positioned at (14,2.1) - adjusted to align bottom
        \includegraphics[height=1.8cm]{../images/logos/logo_reetain.png}
    \end{textblock}
    
    % Add vertical space for the rest of the content
    \vspace{3.5cm}
    
    
    % Title
    {\fontsize{20}{24}\selectfont\textbf{\textcolor{coverblue}{MEMOIRE DE PROJET DE FIN D'ETUDES}}}\\[0.3cm]
    
    % Subtitle
    {\large Pour l'Obtention du Diplôme :}\\[0.3cm]
    
    % Degree title
    {\Large\textbf{\textcolor{coverblue}{INGÉNIEUR D'ÉTAT EN INFORMATIQUE ET INGÉNIERIE DES DONNÉES}}}\\[0.8cm]
    
    % Project title in bordered box with shadow effect using TikZ
    \begin{center}
        \begin{tikzpicture}
            \node[draw=coverblue, thick, fill=white, inner sep=10pt, drop shadow] {
                \begin{minipage}{0.8\textwidth}
                    \begin{center}
                        {\Large\textbf{AMIgo Client Engagement}}\\[0.3cm]
                        {\normalsize Référentiel Client Unifié \& Automatisations Marketing}
                    \end{center}
                \end{minipage}
            };
        \end{tikzpicture}
    \end{center}
    \vspace{0.8cm}
    
    % Information columns with improved formatting - using tabular for alignment
    \begin{center}
        \begin{tabular}{p{0.45\textwidth}p{0.45\textwidth}}
            \hspace*{0pt}\circularbullet~\textbf{\textcolor{coverblue}{Réalisé par :}} & \hspace*{0pt}\circularbullet~\textbf{\textcolor{coverblue}{Effectué à :}} \\
            \hspace{4em}-\hspace*{0.5em} Adnane Idili & \hspace{4em}-\hspace*{0.5em} Reetain Consulting \\
            \hspace{4em}-\hspace*{0.5em} El Mehdi Kamal & \\
            & \\
            \hspace*{0pt}\circularbullet~\textbf{\textcolor{coverblue}{Encadré à l'ENSA par :}} & \hspace*{0pt}\circularbullet~\textbf{\textcolor{coverblue}{Encadré à Reetain par :}} \\
            \hspace{4em}-\hspace*{0.5em} Pr. ATLAS Abdelghafour & \hspace{4em}-\hspace*{0.5em} M. Abdelhay Aboulbaraket \\
        \end{tabular}
    \end{center}
    \vspace{1cm}
    
    % Jury section - Ayoub style
    \begin{center}
        \textbf{\textcolor{coverblue}{Soutenu devant le jury :}}\\[0.5cm]
        Pr. NOM Prénom, Professeur à l'ENSA\\[0.3cm]
        Pr. NOM Prénom, Professeur à l'ENSA\\[0.3cm]
        M. NOM Prénom, Reetain\\[0.3cm]
    \end{center}
    \vspace{0.8cm}
    
    % Academic year
    {\normalsize Année Universitaire : 2024-2025}
\end{center}
\end{titlepage}

% Dédicaces (Dedications)
\chapter*{Dédicaces}
\addcontentsline{toc}{chapter}{Dédicaces}
\begin{center}
    \vspace*{2cm}
    \itshape
    [Votre texte de dédicaces ici]
    \vspace*{2cm}
\end{center}
\clearpage

% Remerciements (Acknowledgements)
\chapter*{Remerciements}
\addcontentsline{toc}{chapter}{Remerciements}
\begin{center}
    \vspace*{1cm}
    \large\textbf{REMERCIEMENTS}
    \vspace*{1cm}
\end{center}

[Votre texte de remerciements ici]

\clearpage

% Table of contents
\tableofcontents
\clearpage

% List of figures
\listoffigures
\clearpage

% List of tables
\listoftables
\clearpage

% Glossary and abbreviations
\chapter*{Glossaire et Abréviations}
\addcontentsline{toc}{chapter}{Glossaire et Abréviations}

\section*{Abréviations}
\begin{tabular}{ll}
    \textbf{PFE} & Projet de Fin d'Études \\
    \textbf{[ABRÉVIATION]} & [DÉFINITION] \\
    \textbf{[ABRÉVIATION]} & [DÉFINITION] \\
    % Ajoutez d'autres abréviations selon vos besoins
\end{tabular}

\section*{Glossaire}
\begin{description}
    \item[\textbf{[TERME]}] [DÉFINITION]
    \item[\textbf{[TERME]}] [DÉFINITION]
    % Ajoutez d'autres termes selon vos besoins
\end{description}

\clearpage

% Executive summary in French
\chapter*{Résumé Exécutif}
\addcontentsline{toc}{chapter}{Résumé Exécutif}

[Votre résumé exécutif en français ici]

\clearpage

% Executive summary in English
\chapter*{Executive Summary}
\addcontentsline{toc}{chapter}{Executive Summary}

[Your executive summary in English here]

\clearpage

% Switch to Arabic numerals for main content
\pagenumbering{arabic}

% Introduction chapter
\chapter{Introduction Générale}
\section{Contexte du Projet}
AMI Paris est une marque de mode prestigieuse qui opère via un réseau de boutiques physiques utilisant le système de gestion Cegid Y2, ainsi que via une plateforme e-commerce basée sur Shopify. Cette structure multicanale, bien que favorable au développement commercial, a engendré des défis significatifs en termes de gestion des données clients.

\section{Problématique}
\begin{mdframed}[backgroundcolor=lightgreen!20, linewidth=1pt]
L'absence d'un référentiel client unique pose plusieurs défis critiques pour AMI Paris, notamment la duplication des données clients entre les systèmes, des processus manuels chronophages pour réconcilier les informations, et une expérience client incohérente entre les canaux physiques et digitaux.
\end{mdframed}

\section{Objectifs}
Le projet AMIgo Client Engagement vise à centraliser et synchroniser toutes les données clients provenant des divers canaux dans Salesforce, permettant une vision client unifiée à 360° essentielle pour améliorer l'expérience client et optimiser les opérations commerciales.

\section{Revue de Littérature / État de l'Art}
Les recherches récentes dans le domaine de la gestion de la relation client (CRM) et de l'intégration des données multicanales dans le secteur du luxe montrent une évolution vers des plateformes unifiées et des approches omnicanales. L'utilisation de l'intelligence artificielle et du machine learning pour la segmentation client et la personnalisation devient également une pratique courante.

\section{Structure du Document}
Ce rapport est organisé selon la structure suivante :
\begin{itemize}
    \item \textbf{Chapitre 1: Contexte Organisationnel et Technique} - Présentation de Reetain Consulting, AMI Paris, et analyse de l'existant
    \item \textbf{Chapitre 2: Fondation du Projet AMIgo} - Architecture technique globale et résultats des premiers sprints
    \item \textbf{Chapitre 3: Référentiel Client Unifié (RCU)} - Conception du modèle de données client et processus GDPR
    \item \textbf{Chapitre 4: Intégration et Chargement des Données} - Framework de chargement initial et intégration
    \item \textbf{Chapitre 5: Expérience Client et Marketing} - Parcours client personnalisés et Marketing Cloud
    \item \textbf{Chapitre 6: Résultats et Évaluation} - Métriques de performance et impact business
\end{itemize}

% Include chapters
% Chapitre 1: Contexte Organisationnel et Technique

\chapter{Contexte Organisationnel et Technique}

\section{Présentation de Reetain Consulting}
\begin{mdframed}[backgroundcolor=lightgreen!20, linewidth=1pt]
Reetain Consulting est une société de conseil spécialisée dans l'implémentation de solutions CRM et l'optimisation de l'expérience client pour les marques de luxe et de mode. Avec une expertise particulière dans l'écosystème Salesforce, Reetain accompagne ses clients dans leur transformation digitale et l'unification de leurs données clients.
\end{mdframed}

\subsection{Expertise et Services}
Reetain Consulting offre une gamme complète de services incluant :

\begin{center}
\begin{tabular}{|>{
\bfseries}p{4cm}|p{9.5cm}|}
\hline
\rowcolor{lightblue} Domaine d'Expertise & Description \\
\hline
Intégration CRM & Implémentation et personnalisation de solutions Salesforce adaptées aux besoins spécifiques des marques de luxe \\
\hline
Stratégie Omnicanale & Conception et mise en œuvre de stratégies d'unification des canaux de vente et de communication \\
\hline
Gestion des Données & Consolidation, nettoyage et enrichissement des données clients provenant de sources multiples \\
\hline
Automatisation Marketing & Développement de parcours client automatisés et personnalisés \\
\hline
\end{tabular}
\end{center}

\section{Présentation d'AMI Paris}
AMI Paris est une marque de mode française fondée en 2011 par Alexandre Mattiussi. Positionnée sur le segment du luxe accessible, AMI propose des collections pour homme et femme qui allient élégance parisienne et décontraction contemporaine.

\subsection{Écosystème Commercial}
La marque opère à travers plusieurs canaux de distribution :

\begin{itemize}
    \item \textbf{Réseau de boutiques physiques} : 12 boutiques en propre réparties dans les capitales mondiales de la mode
    \item \textbf{E-commerce} : Plateforme Shopify représentant 35\% du chiffre d'affaires global
    \item \textbf{Distribution wholesale} : Présence dans plus de 360 points de vente multimarques
    \item \textbf{Marketplaces} : Partenariats avec des plateformes de luxe comme Farfetch et Net-a-Porter
\end{itemize}

\section{Analyse de l'Existant}

\subsection{Systèmes d'Information Actuels}
Avant le projet AMIgo, l'écosystème technique d'AMI Paris présentait une architecture fragmentée :

\begin{center}
\begin{tabular}{|>{
\bfseries}p{3.5cm}|p{5cm}|p{5cm}|}
\hline
\rowcolor{lightblue} Système & Fonction & Limitations \\
\hline
Cegid Y2 & Gestion des points de vente physiques et des clients retail & Données isolées, pas d'intégration native avec l'e-commerce \\
\hline
Shopify & Plateforme e-commerce et gestion des clients online & Base de données clients séparée de Cegid Y2 \\
\hline
Mailchimp & Campagnes email marketing & Synchronisation manuelle des listes, segmentation limitée \\
\hline
Tableur Excel & Réconciliation manuelle des données clients & Processus chronophage et sujet aux erreurs \\
\hline
\end{tabular}
\end{center}

\subsection{Défis de l'Intégration Multicanale}
\begin{mdframed}[backgroundcolor=lightgreen!20, linewidth=1pt]
L'absence d'une vision client unifiée engendre plusieurs problématiques critiques pour AMI Paris :
\begin{itemize}
    \item Impossibilité d'identifier un même client à travers les différents canaux
    \item Duplication des profils clients et des communications
    \item Expérience client incohérente entre les canaux physiques et digitaux
    \item Difficulté à mesurer la valeur client globale et le ROI des actions marketing
    \item Complexité dans la gestion des consentements RGPD à travers les différents systèmes
\end{itemize}
\end{mdframed}

\section{Enjeux Stratégiques du Projet}
Le projet AMIgo Client Engagement s'inscrit dans une stratégie plus large de transformation digitale et d'amélioration de l'expérience client chez AMI Paris.

\subsection{Objectifs Business}
\begin{itemize}
    \item Augmenter le taux de conversion et le panier moyen grâce à une meilleure connaissance client
    \item Améliorer la fidélisation client par des communications personnalisées et pertinentes
    \item Optimiser l'allocation des ressources marketing grâce à une meilleure segmentation
    \item Renforcer la cohérence de l'expérience de marque à travers tous les points de contact
\end{itemize}

\subsection{Indicateurs de Performance Clés}
\begin{center}
\begin{tabular}{|>{
\bfseries}p{5cm}|p{3cm}|p{5cm}|}
\hline
\rowcolor{lightblue} KPI & Objectif & Méthode de Mesure \\
\hline
Taux de conversion omnicanal & +25\% & Suivi des parcours clients cross-canal \\
\hline
Valeur Vie Client (LTV) & +30\% & Analyse de cohortes pré/post implémentation \\
\hline
Taux d'engagement email & +40\% & Métriques d'ouverture et de clic \\
\hline
Efficacité opérationnelle & -50\% de temps & Réduction du temps consacré à la réconciliation des données \\
\hline
\end{tabular}
\end{center}

\section{Méthodologie Adoptée}
Cette section présente l'approche méthodologique adoptée pour mener à bien ce projet.

\subsection{Choix de la Méthodologie}
[Justification du choix de la méthodologie utilisée]

\subsection{Phases du Projet}
[Description des différentes phases prévues pour la réalisation du projet]

\section{Conclusion}
[Conclusion du chapitre résumant les points clés et faisant la transition vers le chapitre suivant]

% Chapitre 2: AMIgo Client Engagement 2.0

\chapter{AMIgo Client Engagement 2.0}

\section{Évolution du Projet AMIgo}

AMIgo Client Engagement a connu une évolution significative depuis sa première version, avec l'introduction d'AMIgo 2.0 qui représente une refonte majeure de l'architecture et des fonctionnalités. Cette nouvelle version répond aux besoins croissants d'AMI Paris en matière de gestion de la relation client et d'automatisation marketing.

\subsection{Bilan d'AMIgo 1.0}

La première version d'AMIgo a permis d'établir les fondations d'un référentiel client unifié, mais présentait certaines limitations :

\begin{itemize}
    \item Synchronisation unidirectionnelle des données (de Cegid Y2 et Shopify vers Salesforce)
    \item Capacités d'automatisation marketing limitées
    \item Absence d'une vue client véritablement omnicanale
    \item Reporting et analyses insuffisants pour les besoins métier
    \item Difficultés dans la gestion des consentements RGPD
\end{itemize}

Ces limitations ont motivé le développement d'AMIgo 2.0, avec pour objectif de créer une plateforme CRM véritablement intégrée et orientée vers l'automatisation.

\subsection{Objectifs d'AMIgo 2.0}

AMIgo 2.0 a été conçu avec plusieurs objectifs stratégiques :

\begin{center}
\begin{tabular}{|>{\bfseries}p{3.5cm}|p{10cm}|}
\hline
\rowcolor{lightblue} Objectif & Description \\
\hline
Intégration bidirectionnelle & Permettre la synchronisation des données dans les deux sens entre Salesforce, Cegid Y2 et Shopify \\
\hline
Automatisation avancée & Développer des parcours client automatisés basés sur les comportements cross-canal \\
\hline
Intelligence client & Implémenter des analyses prédictives pour anticiper les besoins clients \\
\hline
Conformité RGPD renforcée & Centraliser la gestion des consentements avec traçabilité complète \\
\hline
Expérience omnicanale & Offrir une expérience client cohérente à travers tous les points de contact \\
\hline
\end{tabular}
\end{center}

\section{Architecture Technique d'AMIgo 2.0}

L'architecture d'AMIgo 2.0 représente une évolution significative par rapport à la version précédente, avec une approche plus modulaire et orientée API.

\subsection{Vue d'Ensemble de l'Architecture}

\begin{figure}[H]
\centering
\fbox{\parbox{0.9\textwidth}{
    \centering
    \textbf{Architecture AMIgo 2.0}\\[0.5cm]
    \begin{tabular}{|c|c|c|}
    \hline
    \multicolumn{3}{|c|}{\textbf{Couche Présentation}} \\
    \hline
    Salesforce UI & Marketing Cloud UI & Applications Mobiles \\
    \hline
    \multicolumn{3}{|c|}{\textbf{Couche Services}} \\
    \hline
    \multicolumn{1}{|c|}{CRM Core} & \multicolumn{1}{c|}{Marketing Automation} & \multicolumn{1}{c|}{Analytics} \\
    \hline
    \multicolumn{3}{|c|}{\textbf{Couche Intégration}} \\
    \hline
    \multicolumn{1}{|c|}{API Gateway} & \multicolumn{1}{c|}{Event Bus} & \multicolumn{1}{c|}{ETL Services} \\
    \hline
    \multicolumn{3}{|c|}{\textbf{Couche Données}} \\
    \hline
    \multicolumn{1}{|c|}{Salesforce} & \multicolumn{1}{c|}{Cegid Y2} & \multicolumn{1}{c|}{Shopify} \\
    \hline
    \end{tabular}
}}
\caption{Architecture en couches d'AMIgo 2.0}
\label{fig:architecture}
\end{figure}

\subsection{Composants Clés}

\subsubsection{API Gateway}

L'API Gateway joue un rôle central dans AMIgo 2.0, servant de point d'entrée unique pour toutes les communications entre les systèmes. Cette approche présente plusieurs avantages :

\begin{itemize}
    \item Sécurité centralisée avec authentification OAuth 2.0
    \item Gestion du trafic et limitation de débit
    \item Journalisation et surveillance unifiées
    \item Versionnage des API pour une évolution contrôlée
\end{itemize}

\subsubsection{Event Bus}

Le système d'Event Bus est une nouveauté majeure d'AMIgo 2.0, permettant une architecture orientée événements :

\begin{itemize}
    \item Publication et souscription aux événements métier
    \item Déclenchement d'automatisations basées sur les comportements clients
    \item Réduction du couplage entre les systèmes
    \item Amélioration de la résilience et de la scalabilité
\end{itemize}

\subsubsection{Services ETL Améliorés}

Les services ETL (Extract, Transform, Load) ont été considérablement améliorés dans AMIgo 2.0 :

\begin{mdframed}[backgroundcolor=lightblue, linewidth=0pt, innerleftmargin=10pt, innerrightmargin=10pt]
\textbf{Améliorations des Services ETL :}
\begin{itemize}
    \item Traitement en temps réel des données critiques
    \item Synchronisation bidirectionnelle avec gestion des conflits
    \item Validation avancée des données avec correction automatique
    \item Journalisation détaillée pour audit et dépannage
\end{itemize}
\end{mdframed}

\section{Fonctionnalités Innovantes d'AMIgo 2.0}

AMIgo 2.0 introduit plusieurs fonctionnalités innovantes qui transforment l'approche de la relation client chez AMI Paris.

\subsection{Customer Data Platform (CDP)}

La plateforme de données client est au cœur d'AMIgo 2.0, offrant une vue unifiée et enrichie du client :

\begin{itemize}
    \item Profils clients unifiés combinant données transactionnelles et comportementales
    \item Enrichissement des données via des sources tierces
    \item Segmentation dynamique basée sur des attributs multiples
    \item Historique complet des interactions client à travers tous les canaux
\end{itemize}

\subsection{Parcours Client Automatisés}

AMIgo 2.0 permet la création de parcours client sophistiqués :

\begin{center}
\begin{tabular}{|>{\bfseries}p{4cm}|p{9.5cm}|}
\hline
\rowcolor{lightblue} Type de Parcours & Description \\
\hline
Onboarding & Séquence personnalisée pour les nouveaux clients avec recommandations basées sur le premier achat \\
\hline
Réactivation & Parcours ciblant les clients inactifs avec offres personnalisées basées sur l'historique d'achat \\
\hline
Cross-selling & Recommandations automatisées basées sur les achats précédents et le comportement de navigation \\
\hline
VIP & Expérience premium pour les clients à haute valeur avec services exclusifs et offres anticipées \\
\hline
Post-achat & Suivi personnalisé après chaque achat avec conseils d'entretien et suggestions complémentaires \\
\hline
\end{tabular}
\end{center}

\subsection{Intelligence Artificielle et Prédiction}

AMIgo 2.0 intègre des capacités d'intelligence artificielle pour améliorer la compréhension et l'anticipation des besoins clients :

\begin{mdframed}[backgroundcolor=notegreen, linewidth=0pt, innerleftmargin=10pt, innerrightmargin=10pt]
\textbf{Applications de l'IA dans AMIgo 2.0 :}
\begin{itemize}
    \item Prédiction de la valeur vie client (CLV)
    \item Détection des risques d'attrition
    \item Recommandations de produits personnalisées
    \item Optimisation des moments d'engagement
    \item Analyse des sentiments dans les interactions client
\end{itemize}
\end{mdframed}

\section{Gestion des Données et Conformité RGPD}

La gestion des données et la conformité au RGPD sont des aspects fondamentaux d'AMIgo 2.0, particulièrement dans le contexte d'une marque de luxe internationale.

\subsection{Modèle de Données Unifié}

AMIgo 2.0 s'appuie sur un modèle de données unifié qui harmonise les informations provenant de différentes sources :

\begin{figure}[H]
\centering
\fbox{\parbox{0.9\textwidth}{
    \centering
    \textbf{Modèle de Données Simplifié}\\[0.5cm]
    \begin{tabular}{|c|c|c|}
    \hline
    \textbf{Entité} & \textbf{Attributs Clés} & \textbf{Relations} \\
    \hline
    Customer & ID, Profile, Preferences & Orders, Consents, Segments \\
    \hline
    Order & ID, Date, Amount, Channel & Customer, Products, Store \\
    \hline
    Product & ID, Category, Price & Orders, Recommendations \\
    \hline
    Consent & Type, Status, Timestamp & Customer, Communications \\
    \hline
    Interaction & Type, Channel, Timestamp & Customer, Campaign \\
    \hline
    \end{tabular}
}}
\caption{Modèle de données simplifié d'AMIgo 2.0}
\label{fig:datamodel}
\end{figure}

\subsection{Gestion des Consentements}

La gestion des consentements a été entièrement repensée dans AMIgo 2.0 :

\begin{itemize}
    \item Interface centralisée pour la gestion des préférences client
    \item Granularité fine des consentements par canal et type de communication
    \item Historique complet des modifications de consentement avec horodatage
    \item Propagation automatique des changements de consentement vers tous les systèmes
    \item Mécanismes de double opt-in pour les nouveaux abonnements
\end{itemize}

\subsection{Sécurité et Protection des Données}

AMIgo 2.0 implémente des mesures de sécurité avancées pour protéger les données sensibles :

\begin{mdframed}[backgroundcolor=warningorange, linewidth=0pt, innerleftmargin=10pt, innerrightmargin=10pt]
\textbf{Mesures de Sécurité Implémentées :}
\begin{itemize}
    \item Chiffrement des données au repos et en transit
    \item Contrôle d'accès basé sur les rôles (RBAC)
    \item Journalisation complète des accès aux données
    \item Anonymisation des données pour les environnements de test
    \item Processus automatisé pour le droit à l'oubli
    \item Audits de sécurité réguliers
\end{itemize}
\end{mdframed}

\section{Résultats et Impact Commercial}

La mise en œuvre d'AMIgo 2.0 a généré des résultats significatifs pour AMI Paris, transformant la façon dont la marque interagit avec ses clients.

\subsection{Métriques Clés}

Les premiers résultats d'AMIgo 2.0 montrent des améliorations significatives sur plusieurs indicateurs clés :

\begin{center}
\begin{tabular}{|>{\bfseries}p{5cm}|>{\centering}p{3cm}|>{\centering\arraybackslash}p{3cm}|}
\hline
\rowcolor{lightblue} Indicateur & Avant AMIgo 2.0 & Après AMIgo 2.0 \\
\hline
Taux de conversion e-commerce & 2.3\% & 3.8\% \\
\hline
Valeur panier moyen & 420€ & 580€ \\
\hline
Taux d'ouverture des emails & 18\% & 32\% \\
\hline
Taux de réachat à 6 mois & 22\% & 35\% \\
\hline
Temps de résolution service client & 48h & 12h \\
\hline
\end{tabular}
\end{center}

\subsection{Bénéfices Qualitatifs}

Au-delà des métriques quantitatives, AMIgo 2.0 a apporté des bénéfices qualitatifs importants :

\begin{itemize}
    \item Expérience client plus cohérente entre les canaux physiques et digitaux
    \item Meilleure compréhension des parcours d'achat cross-canal
    \item Capacité accrue à anticiper les tendances et les besoins clients
    \item Collaboration renforcée entre les équipes retail et e-commerce
    \item Conformité RGPD simplifiée et plus robuste
\end{itemize}

\section{Perspectives d'Évolution}

AMIgo 2.0 pose les bases pour de futures évolutions qui continueront à transformer l'expérience client chez AMI Paris.

\subsection{Prochaines Étapes}

Plusieurs initiatives sont déjà planifiées pour les prochaines phases de développement :

\begin{mdframed}[backgroundcolor=lightblue, linewidth=0pt, innerleftmargin=10pt, innerrightmargin=10pt]
\textbf{Roadmap AMIgo 3.0 :}
\begin{itemize}
    \item Intégration de capacités de commerce conversationnel via WhatsApp et Instagram
    \item Déploiement d'un système de reconnaissance visuelle pour les recommandations produits
    \item Extension des capacités prédictives avec des modèles de machine learning avancés
    \item Implémentation d'un programme de fidélité omnicanal
    \item Développement d'expériences en réalité augmentée pour le retail
\end{itemize}
\end{mdframed}

\subsection{Vision à Long Terme}

La vision à long terme pour AMIgo s'articule autour de plusieurs axes stratégiques :

\begin{itemize}
    \item Personnalisation hyper-contextuelle basée sur la localisation et le comportement en temps réel
    \item Intégration complète des expériences physiques et digitales
    \item Anticipation proactive des besoins clients grâce à l'intelligence artificielle
    \item Autonomisation des clients dans la gestion de leur relation avec la marque
    \item Création d'une communauté engagée autour des valeurs de la marque
\end{itemize}

\section{Conclusion}

AMIgo 2.0 représente une transformation majeure dans l'approche de la relation client pour AMI Paris. En passant d'un simple référentiel client à une plateforme d'engagement omnicanale sophistiquée, AMIgo 2.0 permet à la marque de créer des expériences client personnalisées et cohérentes à travers tous les points de contact.

Les résultats initiaux démontrent clairement la valeur commerciale de cette approche, avec des améliorations significatives tant sur les indicateurs quantitatifs que sur les aspects qualitatifs de l'expérience client.

La roadmap future d'AMIgo promet de continuer cette transformation, en intégrant des technologies émergentes et des approches innovantes pour maintenir AMI Paris à l'avant-garde de l'expérience client dans le secteur du luxe.

% Uncomment these as they are created
% Chapitre 3: Référentiel Client Unifié (RCU)

\chapter{Référentiel Client Unifié (RCU)}

\section{Conception du Modèle de Données Client}

\begin{mdframed}[backgroundcolor=lightgreen!20, linewidth=1pt]
Le Référentiel Client Unifié (RCU) constitue le cœur du projet AMIgo, permettant de centraliser et d'harmoniser toutes les données clients provenant des différents canaux. Sa conception a nécessité une analyse approfondie des besoins métier et des contraintes techniques pour garantir une vision client à 360°.
\end{mdframed}

\subsection{Architecture du Modèle de Données}

L'architecture du RCU repose sur plusieurs objets Salesforce interconnectés :

\begin{center}
\begin{tabular}{|>{\bfseries}p{4cm}|p{9.5cm}|}
\hline
\rowcolor{lightblue} Objet & Description \\
\hline
Account & Entité centrale représentant le client avec ses informations d'identification et de contact \\
\hline
Store & Représentation des boutiques physiques avec leurs caractéristiques et zones de chalandise \\
\hline
Sales Associate & Profils des conseillers de vente avec leurs attributions et historiques de performance \\
\hline
Consent & Gestion des consentements RGPD avec historisation des modifications \\
\hline
Order & Historique des achats cross-canal avec détails des transactions \\
\hline
\end{tabular}
\end{center}

\subsection{Modélisation des Relations}

Les relations entre les différents objets ont été conçues pour faciliter les requêtes métier tout en maintenant l'intégrité des données :

\begin{itemize}
    \item Relation 1-n entre Account et Order pour l'historique d'achat
    \item Relation n-n entre Account et Store via un objet de jonction pour suivre les préférences de boutique
    \item Relation n-n entre Account et Sales Associate pour la gestion de la clientèle
    \item Relation 1-n entre Account et Consent pour le suivi des consentements marketing
\end{itemize}

\section{CRM RCU - Account Object (ACE-24)}

L'objet Account est la pierre angulaire du RCU, centralisant toutes les informations clients essentielles.

\subsection{Structure de l'Objet Account}

\begin{center}
\begin{tabular}{|>{\bfseries}p{4cm}|p{4cm}|p{5.5cm}|}
\hline
\rowcolor{lightblue} Champ & Type & Description \\
\hline
Customer ID & Texte (External ID) & Identifiant unique du client généré par le système \\
\hline
Email & Email & Adresse email principale du client (unique) \\
\hline
First Name & Texte & Prénom du client \\
\hline
Last Name & Texte & Nom de famille du client \\
\hline
Phone & Téléphone & Numéro de téléphone principal \\
\hline
Preferred Language & Liste de sélection & Langue de communication préférée \\
\hline
Preferred Store & Référence & Boutique préférée du client \\
\hline
Customer Origin & Liste de sélection & Canal d'acquisition (Retail, E-commerce, Wholesale) \\
\hline
Customer Tier & Liste de sélection & Segment client (Standard, Premium, VIP) \\
\hline
\end{tabular}
\end{center}

\subsection{Mécanismes de Déduplication}

\begin{mdframed}[backgroundcolor=lightgreen!20, linewidth=1pt]
La déduplication des profils clients est un enjeu critique pour le RCU. Plusieurs mécanismes ont été implémentés :

\begin{itemize}
    \item Algorithme de matching basé sur l'email, le téléphone et l'adresse postale
    \item Système de scoring de similarité pour les cas ambigus
    \item Processus de validation manuelle pour les cas complexes
    \item Historisation des fusions pour traçabilité
\end{itemize}
\end{mdframed}

\section{CRM RCU - Stores (ACE-10)}

L'objet Store permet de gérer le réseau de boutiques physiques et leurs interactions avec les clients.

\subsection{Structure de l'Objet Store}

\begin{center}
\begin{tabular}{|>{\bfseries}p{4cm}|p{4cm}|p{5.5cm}|}
\hline
\rowcolor{lightblue} Champ & Type & Description \\
\hline
Store ID & Texte (External ID) & Identifiant unique de la boutique \\
\hline
Store Name & Texte & Nom commercial de la boutique \\
\hline
Address & Adresse & Adresse physique complète \\
\hline
Phone & Téléphone & Numéro de contact de la boutique \\
\hline
Email & Email & Adresse email de la boutique \\
\hline
Opening Hours & Texte long & Horaires d'ouverture \\
\hline
Store Manager & Référence & Responsable de la boutique \\
\hline
Region & Liste de sélection & Zone géographique (Europe, Asie, Amériques) \\
\hline
\end{tabular}
\end{center}

\section{CRM RCU - GDPR Process (ACE-13)}

La conformité RGPD est un aspect fondamental du projet AMIgo, nécessitant des mécanismes robustes de gestion des consentements et des droits des clients.

\subsection{Modèle de Consentement}

Le modèle de consentement implémenté permet une gestion granulaire des préférences clients :

\begin{center}
\begin{tabular}{|>{\bfseries}p{4cm}|p{9.5cm}|}
\hline
\rowcolor{lightblue} Type de Consentement & Description \\
\hline
Email Marketing & Consentement pour recevoir des communications marketing par email \\
\hline
SMS Marketing & Consentement pour recevoir des communications marketing par SMS \\
\hline
Téléphone & Consentement pour être contacté par téléphone \\
\hline
Courrier Postal & Consentement pour recevoir des communications par voie postale \\
\hline
Analyse des Données & Consentement pour l'analyse comportementale et la personnalisation \\
\hline
\end{tabular}
\end{center}

\subsection{Processus de Gestion des Droits}

\begin{mdframed}[backgroundcolor=lightgreen!20, linewidth=1pt]
Un workflow automatisé a été implémenté pour traiter les demandes relatives aux droits RGPD :

\begin{itemize}
    \item Droit d'accès : génération automatique d'un rapport des données personnelles
    \item Droit de rectification : interface dédiée pour la mise à jour des informations
    \item Droit à l'oubli : processus d'anonymisation avec validation multi-niveaux
    \item Droit d'opposition : mécanisme de désabonnement avec confirmation
    \item Droit à la portabilité : export des données au format standard
\end{itemize}
\end{mdframed}

\section{CRM RCU - Data Merge Process (ACE-44)}

Le processus de fusion des données est crucial pour maintenir l'intégrité du RCU lors de l'identification de doublons.

\subsection{Algorithme de Fusion}

L'algorithme de fusion implémenté suit une logique de priorisation des sources de données :

\begin{center}
\begin{tabular}{|>{\bfseries}p{3cm}|p{3cm}|p{7.5cm}|}
\hline
\rowcolor{lightblue} Source & Priorité & Justification \\
\hline
Cegid Y2 & Haute & Données collectées en face-à-face avec vérification d'identité \\
\hline
Shopify & Moyenne & Données déclaratives avec validation d'email \\
\hline
Formulaires Web & Basse & Données non vérifiées, potentiellement incomplètes \\
\hline
\end{tabular}
\end{center}

\subsection{Traçabilité des Fusions}

Chaque fusion de profils clients est documentée dans un journal d'audit contenant :

\begin{itemize}
    \item Identifiants des profils source et cible
    \item Horodatage de la fusion
    \item Utilisateur ou processus ayant initié la fusion
    \item Règles de décision appliquées
    \item Champs modifiés avec valeurs avant/après
\end{itemize}

\section{Résultats et Défis}

\subsection{Métriques de Qualité des Données}

\begin{center}
\begin{tabular}{|>{\bfseries}p{5cm}|p{3cm}|p{5cm}|}
\hline
\rowcolor{lightblue} Indicateur & Résultat & Impact Business \\
\hline
Taux de déduplication & 18\% & Réduction des communications en double \\
\hline
Complétude des profils & +45\% & Amélioration de la segmentation client \\
\hline
Précision des données & 92\% & Fiabilité accrue des analyses \\
\hline
Temps de réconciliation & -85\% & Gain de productivité pour les équipes \\
\hline
\end{tabular}
\end{center}

\subsection{Défis Rencontrés et Solutions}

\begin{mdframed}[backgroundcolor=lightgreen!20, linewidth=1pt]
Plusieurs défis techniques et organisationnels ont été surmontés durant l'implémentation du RCU :

\begin{itemize}
    \item \textbf{Normalisation des données} : Développement d'un framework de standardisation des formats d'adresse et de téléphone
    \item \textbf{Gestion des homonymes} : Implémentation d'un système de scoring multi-critères pour différencier les homonymes
    \item \textbf{Résistance au changement} : Programme de formation et d'accompagnement des équipes retail
    \item \textbf{Performance du système} : Optimisation des requêtes et mise en place d'une architecture de cache
\end{itemize}
\end{mdframed}

% Chapitre 4: Intégration et Chargement des Données

\chapter{Intégration et Chargement des Données}

\section{Initial Data Load Framework (ACE-5)}

\begin{mdframed}[backgroundcolor=lightgreen!20, linewidth=1pt]
Le chargement initial des données historiques constitue une étape critique du projet AMIgo, nécessitant une approche méthodique pour garantir l'intégrité et la qualité des données migrées vers le Référentiel Client Unifié.
\end{mdframed}

\subsection{Stratégie de Migration}

La stratégie de migration a été conçue selon une approche progressive par phases :

\begin{center}
\begin{tabular}{|>{\bfseries}p{3cm}|p{5cm}|p{5.5cm}|}
\hline
\rowcolor{lightblue} Phase & Périmètre & Approche \\
\hline
Phase 1 & Données clients Cegid Y2 & Migration complète avec déduplication préalable \\
\hline
Phase 2 & Données clients Shopify & Réconciliation avec les profils existants et création des nouveaux profils \\
\hline
Phase 3 & Historique des achats & Rattachement aux profils clients unifiés \\
\hline
Phase 4 & Données de consentement & Consolidation selon la règle du consentement le plus restrictif \\
\hline
\end{tabular}
\end{center}

\subsection{Framework ETL Personnalisé}

Un framework ETL (Extract, Transform, Load) sur mesure a été développé pour orchestrer le processus de migration :

\begin{itemize}
    \item \textbf{Extraction} : Connecteurs spécifiques pour Cegid Y2 et Shopify avec pagination et gestion des timeouts
    \item \textbf{Transformation} : Pipeline de nettoyage, normalisation et enrichissement des données
    \item \textbf{Chargement} : Mécanisme de chargement par lots avec validation et journalisation
    \item \textbf{Réconciliation} : Algorithmes de matching pour identifier les doublons inter-systèmes
\end{itemize}

\section{Initial Integration Preparation (ACE-4)}

La préparation de l'intégration initiale a impliqué la mise en place des fondations techniques nécessaires à l'interopérabilité des systèmes.

\subsection{Architecture d'Intégration}

\begin{center}
\begin{tabular}{|>{\bfseries}p{4cm}|p{9.5cm}|}
\hline
\rowcolor{lightblue} Composant & Fonction \\
\hline
API Gateway & Point d'entrée unique sécurisé pour toutes les intégrations \\
\hline
MuleSoft ESB & Orchestration des flux de données entre les systèmes \\
\hline
File d'attente & Gestion asynchrone des messages pour garantir la résilience \\
\hline
Service de transformation & Conversion des formats de données entre systèmes \\
\hline
Monitoring & Supervision en temps réel des flux d'intégration \\
\hline
\end{tabular}
\end{center}

\subsection{Sécurisation des Échanges}

\begin{mdframed}[backgroundcolor=lightgreen!20, linewidth=1pt]
La sécurité des données a été une priorité absolue dans la conception de l'architecture d'intégration :

\begin{itemize}
    \item Authentification mutuelle TLS pour tous les échanges inter-systèmes
    \item Chiffrement des données sensibles en transit et au repos
    \item Tokenisation des identifiants clients dans les échanges
    \item Journalisation exhaustive des accès et modifications
    \item Mécanismes de détection d'intrusion et d'anomalies
\end{itemize}
\end{mdframed}

\section{CRM - Data Cloud (ACE-69)}

Salesforce Data Cloud a été implémenté comme couche d'unification et d'enrichissement des données client.

\subsection{Fonctionnalités Implémentées}

\begin{center}
\begin{tabular}{|>{\bfseries}p{4cm}|p{9.5cm}|}
\hline
\rowcolor{lightblue} Fonctionnalité & Description \\
\hline
Identity Resolution & Réconciliation des identités clients à travers les canaux \\
\hline
Data Harmonization & Standardisation et enrichissement automatique des données \\
\hline
Segmentation Avancée & Création de segments dynamiques basés sur comportements et attributs \\
\hline
Activation Omnicanale & Distribution des données client vers les canaux d'activation \\
\hline
Insights en Temps Réel & Génération d'insights actionnables sur les comportements clients \\
\hline
\end{tabular}
\end{center}

\subsection{Modèle de Données Unifié}

Le modèle de données dans Data Cloud a été conçu pour offrir une vue client enrichie :

\begin{itemize}
    \item \textbf{Profil Unifié} : Consolidation des attributs clients provenant de toutes les sources
    \item \textbf{Graphe d'Identité} : Cartographie des identifiants clients à travers les canaux
    \item \textbf{Timeline Comportementale} : Historique chronologique des interactions client
    \item \textbf{Scores Prédictifs} : Indicateurs calculés de propension à l'achat, risque d'attrition, etc.
    \item \textbf{Affinités} : Préférences produits et catégories déduites du comportement
\end{itemize}

\section{CRM - Orders \& Line Items (ACE-70)}

L'intégration des commandes et des lignes de commande permet une vision complète de l'historique d'achat client.

\subsection{Structure des Objets de Transaction}

\begin{center}
\begin{tabular}{|>{\bfseries}p{3.5cm}|p{4cm}|p{6cm}|}
\hline
\rowcolor{lightblue} Objet & Relation & Attributs Clés \\
\hline
Order & Parent de Line Item, enfant d'Account & Date, montant total, canal, devise, statut \\
\hline
Order Line Item & Enfant d'Order, lié à Product & Produit, quantité, prix unitaire, remise \\
\hline
Product & Référencé par Line Item & SKU, catégorie, collection, saison \\
\hline
Payment & Enfant d'Order & Méthode, montant, statut, date \\
\hline
\end{tabular}
\end{center}

\subsection{Synchronisation Bidirectionnelle}

\begin{mdframed}[backgroundcolor=lightgreen!20, linewidth=1pt]
La synchronisation des commandes a été implémentée de manière bidirectionnelle :

\begin{itemize}
    \item \textbf{Cegid Y2 → Salesforce} : Synchronisation en temps réel des achats en boutique
    \item \textbf{Shopify → Salesforce} : Intégration des commandes e-commerce via webhooks
    \item \textbf{Salesforce → Cegid Y2} : Enrichissement des profils clients en boutique
    \item \textbf{Salesforce → Shopify} : Mise à jour des préférences et consentements
\end{itemize}
\end{mdframed}

\section{Défis Techniques et Résolution}

\subsection{Gestion des Volumes de Données}

Le projet a dû faire face à des volumes importants de données historiques :

\begin{center}
\begin{tabular}{|>{\bfseries}p{4cm}|p{3cm}|p{6.5cm}|}
\hline
\rowcolor{lightblue} Type de Données & Volume & Solution Implémentée \\
\hline
Profils clients & 1.2M & Chargement par lots avec fenêtres de maintenance \\
\hline
Transactions & 3.5M & Migration incrémentale par tranches temporelles \\
\hline
Interactions & 8.7M & Archivage des données anciennes avec rétention sélective \\
\hline
Produits & 45K & Synchronisation complète avec gestion des versions \\
\hline
\end{tabular}
\end{center}

\subsection{Résolution des Problèmes d'Intégration}

\begin{mdframed}[backgroundcolor=lightgreen!20, linewidth=1pt]
Plusieurs défis d'intégration ont été surmontés durant les sprints 6-7 :

\begin{itemize}
    \item \textbf{Problème} : Incohérences dans les formats de date entre systèmes\\
      \textbf{Solution} : Normalisation systématique en UTC avec conversion contextuelle
    \item \textbf{Problème} : Latence élevée lors des pics de charge\\
      \textbf{Solution} : Implémentation d'un système de file d'attente prioritaire
    \item \textbf{Problème} : Pertes de connexion avec Cegid Y2\\
      \textbf{Solution} : Mécanisme de retry exponentiel avec circuit breaker
    \item \textbf{Problème} : Doublons de transactions lors des synchronisations\\
      \textbf{Solution} : Système d'idempotence basé sur des identifiants composites
\end{itemize}
\end{mdframed}

% Chapitre 5: Expérience Client et Marketing

\chapter{Expérience Client et Marketing}

\section{CRM - Marketing Cloud (ACE-45)}

\begin{mdframed}[backgroundcolor=lightgreen!20, linewidth=1pt]
L'intégration de Salesforce Marketing Cloud représente un pilier fondamental d'AMIgo 2.0, permettant d'orchestrer des parcours client personnalisés et d'automatiser les communications marketing à travers tous les canaux.
\end{mdframed}

\subsection{Architecture Marketing Cloud}

L'implémentation de Marketing Cloud a été structurée autour de plusieurs modules complémentaires :

\begin{center}
\begin{tabular}{|>{\bfseries}p{4cm}|p{9.5cm}|}
\hline
\rowcolor{lightblue} Module & Fonctionnalité \\
\hline
Email Studio & Création et envoi de communications email personnalisées \\
\hline
Journey Builder & Orchestration des parcours client multi-étapes et multicanaux \\
\hline
Audience Builder & Segmentation dynamique basée sur attributs et comportements \\
\hline
Mobile Studio & Gestion des communications SMS et notifications push \\
\hline
Analytics Builder & Mesure de performance des campagnes et parcours \\
\hline
\end{tabular}
\end{center}

\subsection{Intégration avec le RCU}

L'intégration entre Marketing Cloud et le Référentiel Client Unifié a été réalisée via plusieurs mécanismes :

\begin{itemize}
    \item \textbf{Synchronisation bidirectionnelle} des profils clients et préférences
    \item \textbf{Extension de données} pour enrichir les communications avec les attributs du RCU
    \item \textbf{Partage des événements} pour déclencher des communications contextuelles
    \item \textbf{Unification des consentements} pour garantir la conformité RGPD
    \item \textbf{Consolidation des métriques} d'engagement pour enrichir la vue client
\end{itemize}

\section{CRM - Sales Associate (ACE-85)}

Le module Sales Associate permet d'optimiser l'expérience client en boutique en équipant les conseillers de vente d'outils digitaux connectés au RCU.

\subsection{Fonctionnalités pour les Conseillers de Vente}

\begin{center}
\begin{tabular}{|>{\bfseries}p{4cm}|p{9.5cm}|}
\hline
\rowcolor{lightblue} Fonctionnalité & Description \\
\hline
Client 360° & Accès à la vue complète du client incluant préférences, historique d'achat et interactions \\
\hline
Clienteling & Outils de gestion de la relation client personnalisée et suivi des interactions \\
\hline
Catalogue Produits & Accès au catalogue complet avec disponibilité en temps réel \\
\hline
Prise de Commande & Création de commandes cross-canal avec options de livraison flexibles \\
\hline
Recommandations & Suggestions personnalisées basées sur l'historique et les préférences client \\
\hline
\end{tabular}
\end{center}

\subsection{Application Mobile pour Conseillers}

\begin{mdframed}[backgroundcolor=lightgreen!20, linewidth=1pt]
Une application mobile dédiée aux conseillers de vente a été développée pour faciliter l'accès aux informations client en situation de mobilité :

\begin{itemize}
    \item Interface intuitive optimisée pour tablette
    \item Mode hors-ligne avec synchronisation automatique
    \item Scan de QR code pour identification rapide des clients
    \item Prise de notes et photos avec reconnaissance d'image
    \item Tableau de bord personnalisé par conseiller
\end{itemize}
\end{mdframed}

\section{Parcours Client Personnalisés}

AMIgo 2.0 permet la création et l'orchestration de parcours client sophistiqués basés sur les comportements et préférences individuels.

\subsection{Parcours d'Onboarding}

Le parcours d'onboarding est conçu pour accueillir les nouveaux clients et les familiariser avec l'univers de la marque :

\begin{center}
\begin{tabular}{|>{\bfseries}p{3cm}|p{3cm}|p{7.5cm}|}
\hline
\rowcolor{lightblue} Étape & Timing & Contenu \\
\hline
Bienvenue & J+1 après inscription & Email personnalisé de bienvenue avec présentation de la marque \\
\hline
Découverte & J+3 & Présentation des catégories de produits adaptées au profil \\
\hline
Incitation & J+7 & Offre spéciale premier achat avec code promotionnel \\
\hline
Engagement & J+14 & Invitation à compléter le profil avec préférences \\
\hline
Relation & J+30 & Introduction au conseiller de vente dédié \\
\hline
\end{tabular}
\end{center}

\subsection{Parcours de Réactivation}

\begin{mdframed}[backgroundcolor=lightgreen!20, linewidth=1pt]
Le parcours de réactivation cible les clients inactifs avec une approche progressive :

\begin{itemize}
    \item \textbf{Détection} : Identification automatique des clients sans achat depuis 6 mois
    \item \textbf{Segmentation} : Classification par valeur client historique et motif probable d'inactivité
    \item \textbf{Réengagement} : Séquence de communications personnalisées basées sur les achats précédents
    \item \textbf{Incitation} : Offres spéciales adaptées au profil et à l'historique d'achat
    \item \textbf{Suivi} : Mesure du taux de réactivation et ajustement continu de la stratégie
\end{itemize}
\end{mdframed}

\subsection{Parcours Cross-selling}

Le parcours de cross-selling vise à élargir le panier moyen et la diversité des catégories achetées :

\begin{center}
\begin{tabular}{|>{\bfseries}p{4cm}|p{9.5cm}|}
\hline
\rowcolor{lightblue} Mécanisme & Fonctionnement \\
\hline
Recommandations produits & Suggestions automatisées basées sur l'analyse des achats précédents et comportements similaires \\
\hline
Compléments de look & Présentation d'articles complémentaires aux achats récents \\
\hline
Nouvelles catégories & Introduction progressive à des catégories non encore explorées par le client \\
\hline
Événements exclusifs & Invitations à des événements liés à des catégories complémentaires \\
\hline
\end{tabular}
\end{center}

\subsection{Parcours VIP}

Le parcours VIP offre une expérience premium aux clients à haute valeur :

\begin{itemize}
    \item \textbf{Identification} : Détection automatique des clients VIP selon critères prédéfinis (LTV, fréquence, récence)
    \item \textbf{Services exclusifs} : Accès prioritaire, personal shopper, livraison express gratuite
    \item \textbf{Avant-premières} : Accès anticipé aux nouvelles collections et ventes privées
    \item \textbf{Événements} : Invitations personnalisées à des événements exclusifs
    \item \textbf{Reconnaissance} : Attentions spéciales pour les occasions importantes (anniversaire, fidélité)
\end{itemize}

\section{Résultats du Sprint 8 et Perspectives}

\subsection{Réalisations Clés du Sprint 8}

\begin{center}
\begin{tabular}{|>{\bfseries}p{5cm}|p{8.5cm}|}
\hline
\rowcolor{lightblue} Livrable & Impact \\
\hline
RCU Init Load & Chargement initial réussi des objets Account, Store et Sales Associate \\
\hline
Marketing Cloud Connect & Implémentation et configuration de la connexion bidirectionnelle \\
\hline
IP Warming & Lancement du plan de réchauffement d'IP pour optimiser la délivrabilité \\
\hline
Unsubscribe Page & Mise en place d'une page de désabonnement conforme RGPD \\
\hline
1ère Campagne Email & Déploiement de la première campagne test avec segmentation avancée \\
\hline
\end{tabular}
\end{center}

\subsection{Métriques d'Engagement Initial}

\begin{mdframed}[backgroundcolor=lightgreen!20, linewidth=1pt]
Les premières campagnes déployées via Marketing Cloud ont montré des résultats prometteurs :

\begin{itemize}
    \item \textbf{Taux d'ouverture} : 32\% (vs 18\% avec la solution précédente)
    \item \textbf{Taux de clic} : 4.8\% (vs 2.1\% précédemment)
    \item \textbf{Taux de désabonnement} : 0.3\% (vs 0.8\% précédemment)
    \item \textbf{Taux de conversion} : 2.7\% (vs 1.2\% précédemment)
    \item \textbf{Engagement cross-canal} : 15\% des clients online ont visité une boutique dans les 7 jours
\end{itemize}
\end{mdframed}

\subsection{Prochaines Étapes}

Les développements prévus pour les prochains sprints incluent :

\begin{itemize}
    \item Finalisation des parcours client automatisés (Welcome, Winback, Lapsed)
    \item Déploiement complet des templates email responsifs
    \item Intégration des données transactionnelles dans les parcours
    \item Mise en place des campagnes SMS pour les clients opt-in
    \item Développement des tableaux de bord d'analyse de performance
\end{itemize}

% Chapitre 6: Résultats et Évaluation

\chapter{Résultats et Évaluation}

\section{Métriques de Performance Technique}

\begin{mdframed}[backgroundcolor=lightgreen!20, linewidth=1pt]
L'évaluation des performances techniques du projet AMIgo Client Engagement s'appuie sur un ensemble d'indicateurs quantitatifs permettant de mesurer l'efficacité et la robustesse de la solution implémentée.
\end{mdframed}

\subsection{Performance du Système}

Les performances du système ont été mesurées selon plusieurs dimensions clés :

\begin{center}
\begin{tabular}{|>{\bfseries}p{4cm}|p{3cm}|p{6.5cm}|}
\hline
\rowcolor{lightblue} Indicateur & Résultat & Objectif Initial \\
\hline
Temps de réponse moyen & 280ms & <500ms \\
\hline
Disponibilité système & 99.97\% & >99.9\% \\
\hline
Débit de synchronisation & 120 tx/sec & >100 tx/sec \\
\hline
Taux d'erreur intégration & 0.03\% & <0.1\% \\
\hline
Temps de récupération & <3 min & <5 min \\
\hline
\end{tabular}
\end{center}

\subsection{Qualité des Données}

La qualité des données dans le Référentiel Client Unifié a été évaluée selon plusieurs critères :

\begin{itemize}
    \item \textbf{Complétude} : 92\% des champs obligatoires renseignés (vs 68\% avant AMIgo 2.0)
    \item \textbf{Exactitude} : 97\% de données validées conformes (adresses, emails, téléphones)
    \item \textbf{Cohérence} : Réduction de 98\% des incohérences inter-systèmes
    \item \textbf{Unicité} : Élimination de 99.5\% des doublons identifiés
    \item \textbf{Fraîcheur} : 95\% des modifications synchronisées en moins de 5 minutes
\end{itemize}

\section{Impact sur l'Expérience Client}

\subsection{Amélioration des Parcours Client}

\begin{center}
\begin{tabular}{|>{\bfseries}p{4cm}|p{3cm}|p{6.5cm}|}
\hline
\rowcolor{lightblue} Parcours & Amélioration & Impact Business \\
\hline
Onboarding & +45\% & Augmentation du taux de premier achat \\
\hline
Cross-canal & +68\% & Hausse des clients actifs sur plusieurs canaux \\
\hline
Réactivation & +32\% & Amélioration du taux de rétention client \\
\hline
VIP & +27\% & Croissance du panier moyen des clients premium \\
\hline
\end{tabular}
\end{center}

\subsection{Métriques d'Engagement Client}

\begin{mdframed}[backgroundcolor=lightgreen!20, linewidth=1pt]
Les métriques d'engagement client montrent une amélioration significative suite à l'implémentation d'AMIgo 2.0 :

\begin{itemize}
    \item \textbf{Taux d'ouverture email} : 32\% vs 18\% précédemment (+77\%)
    \item \textbf{Taux de clic} : 4.8\% vs 2.1\% précédemment (+129\%)
    \item \textbf{Taux de conversion des campagnes} : 2.7\% vs 1.2\% précédemment (+125\%)
    \item \textbf{Engagement cross-canal} : 15\% des clients online visitent une boutique dans les 7 jours
    \item \textbf{Satisfaction client (NPS)} : 72 vs 58 précédemment (+24\%)
\end{itemize}
\end{mdframed}

\section{Bénéfices pour les Équipes Métier}

\subsection{Impact sur les Opérations Marketing}

L'implémentation d'AMIgo 2.0 a transformé les pratiques des équipes marketing :

\begin{center}
\begin{tabular}{|>{\bfseries}p{4cm}|p{9.5cm}|}
\hline
\rowcolor{lightblue} Domaine & Amélioration \\
\hline
Segmentation & Passage de 8 segments statiques à plus de 50 segments dynamiques \\
\hline
Personnalisation & Capacité à personnaliser 15 variables de contenu vs 3 précédemment \\
\hline
Réactivité & Réduction du temps de conception de campagne de 2 semaines à 3 jours \\
\hline
Analyse & Tableaux de bord unifiés avec métriques cross-canal en temps réel \\
\hline
Test & Capacité à réaliser des tests A/B automatisés avec optimisation continue \\
\hline
\end{tabular}
\end{center}

\subsection{Impact sur les Équipes Retail}

\begin{itemize}
    \item \textbf{Productivité des conseillers} : +35\% de temps consacré à la vente vs tâches administratives
    \item \textbf{Connaissance client} : Accès instantané à 100\% de l'historique client vs 40\% précédemment
    \item \textbf{Vente additionnelle} : +28\% de ventes complémentaires grâce aux recommandations
    \item \textbf{Communication} : Réduction de 65\% des communications redondantes avec les clients
    \item \textbf{Service client} : Résolution au premier contact améliorée de 72\% à 94\%
\end{itemize}

\section{Retour sur Investissement}

\subsection{Analyse Coûts-Bénéfices}

\begin{mdframed}[backgroundcolor=lightgreen!20, linewidth=1pt]
L'analyse coûts-bénéfices du projet AMIgo 2.0 démontre un retour sur investissement significatif :

\begin{itemize}
    \item \textbf{Investissement total} : 450K€ (licences, développement, formation)
    \item \textbf{Économies opérationnelles annuelles} : 180K€ (réduction des tâches manuelles)
    \item \textbf{Augmentation du chiffre d'affaires} : +3.8\% global, soit environ 950K€ annuels
    \item \textbf{ROI calculé} : 247\% sur 3 ans
    \item \textbf{Point d'équilibre} : Atteint en 14 mois
\end{itemize}
\end{mdframed}

\subsection{Bénéfices Intangibles}

Au-delà des métriques quantifiables, plusieurs bénéfices intangibles ont été identifiés :

\begin{center}
\begin{tabular}{|>{\bfseries}p{4cm}|p{9.5cm}|}
\hline
\rowcolor{lightblue} Bénéfice & Description \\
\hline
Image de marque & Perception renforcée d'AMI Paris comme marque innovante et attentive \\
\hline
Agilité organisationnelle & Capacité accrue à s'adapter rapidement aux évolutions du marché \\
\hline
Culture data-driven & Développement d'une culture de décision basée sur les données \\
\hline
Conformité RGPD & Réduction significative des risques liés à la protection des données \\
\hline
Capital de connaissances & Constitution d'une base de connaissances client stratégique \\
\hline
\end{tabular}
\end{center}

\section{Comparaison avec les Objectifs Initiaux}

\subsection{Réalisation des Objectifs}

\begin{center}
\begin{tabular}{|>{\bfseries}p{5cm}|p{2.5cm}|p{6cm}|}
\hline
\rowcolor{lightblue} Objectif Initial & Statut & Commentaire \\
\hline
Intégration bidirectionnelle & Réalisé & Synchronisation complète entre Salesforce, Cegid Y2 et Shopify \\
\hline
Automatisation avancée & Réalisé & Parcours client automatisés implémentés avec succès \\
\hline
Intelligence client & Partiellement & Analyses prédictives en phase finale de développement \\
\hline
Conformité RGPD renforcée & Réalisé & Gestion centralisée des consentements opérationnelle \\
\hline
Expérience omnicanale & Réalisé & Cohérence établie à travers tous les points de contact \\
\hline
\end{tabular}
\end{center}

\subsection{Écarts et Ajustements}

\begin{mdframed}[backgroundcolor=lightgreen!20, linewidth=1pt]
Certains écarts par rapport aux objectifs initiaux ont nécessité des ajustements :

\begin{itemize}
    \item \textbf{Écart} : Complexité sous-estimée de l'intégration avec Cegid Y2\\
      \textbf{Ajustement} : Développement d'un connecteur spécifique et extension du calendrier
    \item \textbf{Écart} : Volume de données historiques supérieur aux prévisions\\
      \textbf{Ajustement} : Optimisation de l'architecture de stockage et stratégie d'archivage
    \item \textbf{Écart} : Adoption plus lente que prévue par les équipes retail\\
      \textbf{Ajustement} : Programme de formation renforcé et désignation d'ambassadeurs
    \item \textbf{Écart} : Besoins additionnels en rapports personnalisés\\
      \textbf{Ajustement} : Développement d'un framework de reporting flexible
\end{itemize}
\end{mdframed}

\section{Enseignements et Bonnes Pratiques}

\subsection{Facteurs Clés de Succès}

Les principaux facteurs ayant contribué au succès du projet incluent :

\begin{itemize}
    \item \textbf{Sponsorship exécutif} fort et implication continue de la direction
    \item \textbf{Approche agile} avec livraisons incrémentales et feedback continu
    \item \textbf{Équipe pluridisciplinaire} combinant expertise technique et métier
    \item \textbf{Focus sur la qualité des données} dès les premières phases du projet
    \item \textbf{Communication transparente} sur les objectifs et avancées du projet
\end{itemize}

\subsection{Leçons Apprises}

\begin{center}
\begin{tabular}{|>{\bfseries}p{4cm}|p{9.5cm}|}
\hline
\rowcolor{lightblue} Domaine & Leçon Apprise \\
\hline
Intégration & Importance de tests d'intégration exhaustifs avec volumes réels \\
\hline
Gestion du changement & Nécessité d'impliquer les utilisateurs finaux dès la phase de conception \\
\hline
Architecture & Bénéfices d'une approche modulaire facilitant les évolutions futures \\
\hline
Gouvernance des données & Importance critique des règles de gouvernance dès le départ \\
\hline
Performance & Nécessité de monitorer les performances en conditions réelles \\
\hline
\end{tabular}
\end{center}


% Conclusion chapter
\chapter{Conclusion Générale}
\section{Synthèse des Travaux}
[Synthèse des travaux ici]

\section{Contributions}
[Contributions ici]

\section{Perspectives}
[Perspectives ici]

% Bibliography
\bibliographystyle{plainnat}
\bibliography{latex/references}
\addcontentsline{toc}{chapter}{Bibliographie}

% Appendices
\begin{appendices}
\chapter{Annexe A: [TITRE DE L'ANNEXE]}
[Contenu de l'annexe A ici]

\chapter{Annexe B: [TITRE DE L'ANNEXE]}
[Contenu de l'annexe B ici]
\end{appendices}

\end{document}
